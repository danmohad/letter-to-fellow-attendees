\documentclass[a4paper,12pt]{article}
\usepackage{fancyhdr}
\usepackage{lastpage}


\pagestyle{fancy}
\fancyhf{}
\fancyhead[L]{Letter to fellow attendees}
\fancyhead[R]{D. Mohaddes}
\fancyfoot[C]{\thepage/\pageref{LastPage}}

\begin{document}

\noindent \today\\

\noindent Esteemed fellow attendees of NeurIPS 2024,\\

\noindent My name is Danyal Mohaddes, and I am a nobody in AI/ML research. My views are my own and do not represent those of my employer nor any institution I have had the honor of attending.

Words have power. Deep, long-lasting power that spans space and time. It is therefore incumbent upon us to clearly and precisely rebuke the words of our distinguished speaker, Professor Rosalind Picard, in regard to our Chinese colleagues. With all due respect to the conference organizers, such a rebuke must come from us, the attendees, as we constitute the audience of that talk. I have found the confidence to do so because of the bravery of the Chinese colleague who raised her voice at the end of the question-and-answer period. I can only hope to expand upon what she said in the moment, with the benefit of video playback. I will not directly quote the distinguished speaker nor the brave colleague here, but instead humbly request that any reader of my words first listen to the talk from 40:03 to 41:29 regarding the relevant slide, and from 1:04:41 to the end for the relevant question and its answer, to have complete context.

\paragraph{The Myth}
There is a pervasive, long-standing, racist myth in the upper echelons of the academic community, the community to which our distinguished speaker and many of our fellow attendees belong, that Chinese students and institutions produce research of inherently inferior quality, that Chinese students cheat their way into Western academic institutions and steal the work of others, that Chinese academics collude to take over journal and conference editorial boards, play favorites to achieve acceptance for their publications, and that these things taken together mean that Chinese students and academics must be quietly excluded when possible, and otherwise uncomfortably tolerated. Much like antisemitism, this myth is typically unspoken, expressing itself predominantly in insidious ways, as unexpectedly rejected papers, atypical levels of scrutiny for gaining acceptance to and graduating from institutions, and most crucially, of having one’s every success and achievement looked at as a probable grounds for culpability in an ethno-national conspiracy.

Let us be frank. Anyone who has walked the halls of our institutions is familiar with this myth. I should like to hope that none of my fellow attendees subscribe to it, that we can agree that the pervasiveness of this myth is a source shame for all academia and institutions of higher learning, and that though we are quiet in our words, in our hearts we condemn it and feel solidarity for our colleagues. I am certain that we are all disturbed in small settings when a colleague, particularly one in a position of authority, promulgates a myth against any of our fellow researchers because of their ethnic or national origin, and that when we hear it, we all quietly promise ourselves that one day when we reach their position, we will be different.

Today, however, we sadly do not have the luxury of that quiet contemplation, for our distinguished speaker has spoken the quiet myth out loud, and has done so as the final invited speaker at our field’s most prestigious forum, a field that has the honor today of being in the spotlight of the world. Regardless of how we may feel, the speaker’s words represent us collectively, and the world cannot hear what we say in our hearts. Nor can our Chinese colleagues. It therefore behooves us to systematically deconstruct and rebuke our distinguished speaker’s comments. Apologies from and punishments of the speaker are irrelevant to my present discussion. The point is that we must reflect on, rebuke and defeat flawed thinking with clear, intellectual arguments. That is what we, as researchers, are best at.

\paragraph{Deconstruction} Let us suppose that our speaker’s anecdote is indeed true: that, per her own affirmation, in her entire academic career, a single Chinese student claimed, while facing expulsion, that they were not culpable for academic dishonesty because they were not taught “morals or values.” I am certain that as members of a field derived in part from applied statistics, we can appreciate that a single anecdote is not representative of millions of pupils and graduates. Furthermore, as individuals who have attended tertiary institutions, and generally as adult humans who have walked the earth, we can recognize that a student facing expulsion from a university, and likely from the country soon after due to visa dependence on continued attendance, would say quite literally any combination of words they think might result in reduced punishment. A witness under duress is no reliable witness, much less a spokesperson for an entire nation and its education system.

Outside of the distinguished speaker’s comments on the issue of the aforementioned expulsion, the speaker demonstrated what we may describe as arrogance and hubris in regard to Chinese academic institutions. Her willingness to take at face value the expelled student’s misguided defense of his or her actions as a potential indictment of the entirety of Chinese academia can only be viewed as a blatant instance of confirmation bias, and in the context of the aforementioned pervasive myth, we may infer that this is an ethno-racially motivated bias. Again, as students of statistics, I am sure my colleagues can appreciate the importance of a prior. This aspect of the distinguished speaker’s statements is particularly egregious given the context of our conference: a cursory examination of poster sessions and oral presentations is sufficient to observe that a plurality of the authors and co-authors are of Chinese origin. I do not have access to precise statistics, but I invite my skilled colleagues to run the numbers. The distinguished speaker’s comments would have us question the morality and integrity of all our Chinese colleagues, denigrating their achievement in having their papers selected for presentation in our prestigious forum. This is unacceptable. 

One unspoken aspect of the context of the distinguished speaker’s comments is that this insult our Chinese colleagues suffered was likely not the first, but the second in attending our conference. How many of our colleagues were not in attendance to present their own work? We politely refer to “visa issues,” but we know that many of our colleagues are made to overcome extraordinary hurdles to attend. Yet our colleagues take this in stride, and do us the honor of undertaking this process and traveling to present their research. It is in this context that the distinguished speaker saw fit to question their morality and integrity.

\paragraph{Conclusion}Of course, none of the foregoing is particular to our Chinese colleagues. Many are impacted by the visa issue, and such myths and stereotypes abound; we need not list them all here. We are researchers, not politicians, and it is not our domain of expertise to attempt to cure society of all its ills. Indeed, I must apologize to my fellow attendees for polluting their attention with talk that is wholely non-technical in nature and not at all representative of the otherwise incredible conference and exhibition we have had the priviledge of attending, with due thanks to the tireless organizers. I believe that as researchers, it is in our nature to brush off awkward and painful moments like this and carry on with our beloved task of knowledge creation. But when someone stands before us, in our forum, in the home for the work we choose to dedicate our lives to, pontificating about ethics, and then insults and promulgates a harmful myth against thousands of our colleagues, then we must not stand for it. And so I humbly submit this letter as a rebuke to the words of Professor Rosalind Picard, and I hope that my words reflect some of the thoughts and feelings of my fellow attendees.
\\ \\
\noindent \textbf{TL;DR} Let the old poem never be heard anew: First they came for the Chinese, but I did not speak, for I was not Chinese.
\\ \\
\noindent Yours most humbly and sincerely,\\
\noindent Danyal Mohaddes

\end{document}
